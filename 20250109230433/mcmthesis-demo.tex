%% This is file `mcmthesis-demo.tex',
%% generated with the docstrip utility.
%%
%% The original source files were:
%%
%% mcmthesis.dtx  (with options: `demo')
%% 
%% -----------------------------------
%% This is a generated file.
%% 
%% Copyright (C) 2010 -- 2015 by latexstudio
%%       2014 -- 2019 by Liam Huang
%%       2019 -- present by latexstudio.net
%% 
%% This work may be distributed and/or modified under the
%% conditions of the LaTeX Project Public License, either version 1.3
%% of this license or (at your option) any later version.
%% The latest version of this license is in
%%   http://www.latex-project.org/lppl.txt
%% and version 1.3 or later is part of all distributions of LaTeX
%% version 2005/12/01 or later.
%% 
%% The Current Maintainer of this work is latexstudio.net.
%% 
%% !Mode:: "TeX:UTF-8"
\documentclass{mcmthesis}
 %\documentclass[CTeX = true]{mcmthesis}  % 当使用 CTeX 套装时请注释上一行使用该行的设置
\mcmsetup{tstyle=\color{red}\bfseries,%修改题号,队号的颜色和加粗显示,黑色可以修改为 black
        tcn = 2525570, problem = B, %修改队号,参赛题号
        sheet = true, titleinsheet = true, keywordsinsheet = true,
        titlepage = false, abstract = true}

  %四款字体可以选择
  %\usepackage{times}
  %\usepackage{newtxtext}
  %\usepackage{palatino}
 \usepackage{txfonts}

\usepackage{booktabs}
\usepackage{float}
\usepackage{indentfirst}  %首行缩进,注释掉,首行就不再缩进。
\usepackage{lipsum}
\usepackage{hyperref}
\title{The title of Problem B (To be revised)}
\author{\small \href{https://www.latexstudio.net/}
  {\includegraphics[width=7cm]{mcmthesis-logo}}}
\date{\today}
\begin{document}
\begin{abstract}
\par 
Use this template to begin typing the first page (summary page) of your electronic report. This template uses a 12-point Times New Roman font. Submit your paper as an Adobe PDF electronic file (e.g. 1111111.pdf), typed in English, with a readable font of at least 12-point type.

Do not include the name of your school, advisor, or team members on this or any page.

Papers must be within the 25 page limit.

Be sure to change the control number and problem choice above.
You may delete these instructions as you begin to type your report here.

(We will modify this page when this paper is almost done.)
\begin{keywords}
keyword1; keyword2
\end{keywords}
\end{abstract}
\maketitle
%% Generate the Table of Contents, if it's needed.
\tableofcontents
\newpage
%%
%% Generate the Memorandum, if it's needed.
%% \memoto{\LaTeX{}studio}
%% \memofrom{Liam Huang}
%% \memosubject{Happy \TeX{}ing!}
%% \memodate{\today}
%% \memologo{\LARGE I'm pretending to be a LOGO!}
%% \begin{memo}[Memorandum]
%%   \lipsum[1-3]
%% \end{memo}
%%
\section{Introduction}
\subsection{Problem Background}
% 有一半多一点的优秀论文这里会放一张没什么信息量的那种图,
% 因为我们这里包含了旅游业带来的经济增长、环境破坏和居民态度三个部分,
% 不是很好总结,就先没有放图了
Juneau, Alaska, a city with approximately 30,000 residents, has witnesseda dramatic boom in the tourism, which reached a peak of 1.6 million passengersin 2023. 
This influx of vistors has introduced significant economic benefits, generating about \$375 million in revenue for the city. 
However, it has also brought challenges like overcrowding and environmental concerns.

One of the most urgent issues is the rapid recession of Mendenhall Glacier, a premier attraction that has receded by around eight football fields since 2007. 
This retreat is partly attributed to warming temperatures, which are exacerbated by the increased human activity associated with overtourism. 
Moreover, the hidden costs of tourism, including pressure on local infrastructure and an overall increased carbon footprint, poses serious challenges to the environmentally sensitive regions. 
Thought various measures have been taken to ease the burden, like increased hotel taxes, visitor fees and restrictions on alcohol comsumption, no tangible results have yet been seen.

While numerous locals who rely on tourism prefer to see growing number benefit their businesses, many others are frustrated and are either leaving or protesting against the impact of tourists. 
Therefore, how to realize the sustainability of the tourism becomes a great challenge we need to address.

\subsection{Restatement of the Problem}
Considering the background information and restricted conditions identified in the problem statement, we are required to solve the following problems:
\begin{itemize}
  \item \textbf{Problem 1: Model Development} \\
    Develop a model to stabilize Juneau's tourism industry by optimizing income while controlling the carbon footprint and rate of melting of glaciers. 
    Include a sensitivity analysis to identify the most significant factors.
    
  \item \textbf{Problem 2: Revenue Allocation} \\
    Allocate the expenditures from additional revenue to areas such as infrastructure and environmental protection reasonably so that the social benifits are maxmized.
    
  \item \textbf{Problem 3: Model Adaptability} \\
    Adapt the model to other overtourism-affected tourist destinations, showing how location-specific factors influence measure importance. 
    Use the model establisted above to promote less-visited locations for better balance.
    
  \item \textbf{Problem 4: Memo to Tourist Council} \\
    Draft a one-page memo outlining predictions, the effects of various measures, and suggestions for optimizing outcomes.
\end{itemize}


\subsection{Our Work}
%TODO

\section{Assumption and Justification}
\begin{itemize}
  \item \textbf{Assumption1:} The data we use are accurate and valid. 
  \item \textbf{Justification1:} Our data is collected from the Juneau government and some other official websites and research papers. 
  It is reasonable to assume that the data are of high quality.

  \item \textbf{Assumption2:} Juneau will remain relatively stable, with no drastic environmental changes or social unrest.
  \item \textbf{Justification2:} A stable natural and social environment provides a predictable framework within which we can build model and make decisions. 
  It is important to note that this assumption does not neglect the potential impact of predictable envirmental and social change.
  
  \item \textbf{Assumption3:} The tourists and locals are are rational decision-makers aiming to maximize their own utility and possessing complete logical reasoning abilities.% 因为用了博弈论模型,加上理性人假设
  \item \textbf{Justification3:} This assumption is well-grounded in economic theory. Tourists typically make travel decisions based on cost-benefit analyses\cite{crouch2004determinants}, 
  and local residents engage in tourism-related activities to maximize their benefits\cite{ap1992residents}.
\end{itemize}

\section{Notations}
\begin{table}[H]
  \caption{\textbf{Notations}}
  \centering
  \begin{tabular}{ccc}
    \toprule
      Symbol & Definition & Unit \\
    \midrule
      $V$ & annual number of tourists & million \\ 
      $C_{v}$ & annual carbon emissions per tourist & ton/person \\
      $W_v$ & annual water consumption per tourist & $m^3$/person \\
      $F_v$ & annual amount of waste produced per tourist & tons/person \\
      $R_v$ & income from every tourist's comsumption annually & \$/person \\
      $T_h$ & hotel tax rate & \% \\
      $T_v$ & tourist fees & \$/person \\
      $G_r$ & rate of retreat of glaciers & meter/year \\
      $E$ & comprehensive indicators of environmental quality & / \\
      $F$ & annual amount of waste generated & ton/year \\
      $K$ & infrastructure capital & / \\
      \bottomrule
  \end{tabular}
\end{table}

\section{Dynamic Management Model of Sustainable Tourism Based on Tripartite Game}
\subsection{Data Collection}
\begin{table}[H]
  \caption{\textbf{Data Websites}} 
  \centering
  \begin{tabular}{cc}
    \toprule 
      Database Name & Database Website \\
    \midrule
      Juneau & \href{https://juneau.org}{https://juneau.org} \\ 
      JEDC & \href{https://www.jedc.org}{https://www.jedc.org} \\
      EJSCREEN & \href{https://gaftp.epa.gov/ejscreen/}{https://gaftp.epa.gov/ejscreen/} \\ 
      AntarcticGlaciers & \href{https://www.antarcticglaciers.org}{https://www.antarcticglaciers.org} \\
      GHGRP & \href{https://www.epa.gov/ghgreporting}{https://www.epa.gov/ghgreporting} \\
    \bottomrule
  \end{tabular}
\end{table}
\subsection{Data Preprocessing}

\section{The Model Results}

\section{Validating the Model}

\section{Conclusions}

\section{A Summary}

\section{Evaluate of the Mode}

\section{Strengths and weaknesses}

\subsection{Strengths}


% \subsection{How to cite?}
% bibliography cite use \cite{1,2,3}

% AI cite use \AIcite{AI1,AI2,AI3}

% \begin{thebibliography}{99}
% \bibitem{1} D.~E. KNUTH   The \TeX{}book  the American
% Mathematical Society and Addison-Wesley
% Publishing Company , 1984-1986.
% \bibitem{2}Lamport, Leslie,  \LaTeX{}: `` A Document Preparation System '',
% Addison-Wesley Publishing Company, 1986.
% \bibitem{3}\url{https://www.latexstudio.net/}
% \end{thebibliography}
\bibliographystyle{unsrt}
\bibliography{mcmthesis-demo}

\begin{appendices}

\section{First appendix}

In addition, your report must include a letter to the Chief Financial Officer (CFO) of the Goodgrant Foundation, Mr. Alpha Chiang, that describes the optimal investment strategy, your modeling approach and major results, and a brief discussion of your proposed concept of a return-on-investment (ROI). This letter should be no more than two pages in length.

\begin{letter}{Dear, Mr. Alpha Chiang}

\lipsum[1-2]

\vspace{\parskip}

Sincerely yours,

Your friends

\end{letter}
Here are simulation programmes we used in our model as follow.\\

\textbf{\textcolor[rgb]{0.98,0.00,0.00}{Input matlab source:}}
\lstinputlisting[language=Matlab]{./code/mcmthesis-matlab1.m}

\section{Second appendix}

some more text \textcolor[rgb]{0.98,0.00,0.00}{\textbf{Input C++ source:}}
\lstinputlisting[language=C++]{./code/mcmthesis-sudoku.cpp}

\end{appendices}


\AImatter

\begin{ReportAiUse}{9}
\bibitem{AI1}
Bing AI\\
Query1: write a sort algorithm\\
Output: Sure, I can help you with that. Here is a simple implementation of the selection sort algorithm in Python:
\begin{lstlisting}[language=python]
def selection_sort(arr):
    for i in range(len(arr)):
        min_idx = i
        for j in range(i + 1, len(arr)):
            if arr[min_idx] > arr[j]:
                min_idx = j
        arr[i], arr[min_idx] = arr[min_idx], arr[i]
    return arr
\end{lstlisting}
This algorithm sorts an array by repeatedly finding the minimum element from the unsorted part of the array and putting it at the beginning. \href{https://www.geeksforgeeks.org/sorting-algorithms/}{The time complexity of this algorithm is O($n^2$)}.

There are many other sorting algorithms available such as bubble sort, insertion sort, merge sort, quick sort, heap sort, and more. Each of these algorithms has its own advantages and disadvantages. You can learn more about them on GeeksforGeeks and Programiz.

I hope this helps!

\bibitem{AI2}
OpenAl Enie (Nov 5, 2023 version, Ernie 4.0)\\
Query1: <insert the exact wording of any subsequent input into the Al tool>\\
Output: <insert the complete output from the second query>

\bibitem{AI3}
Github CoPilot (Feb 3, 2024 version)\\
Query1: <insert the exact wording you input into the Al tool>\\
Output: <insert the complete output fron the Al tool>

\bibitem{AI4}
Google Bard (Feb 2,2024 version)\\
Query: <insert the exact wording of your query>\\
Output: <irsert the complete output from the Al tool>

\bibitem{AI5}
Bing AI\\
Query: write a FFT algorithm\\
Output: Sure, here is a Python implementation of the Fast Fourier Transform (FFT) algorithm:
\begin{lstlisting}[language=python]
import numpy as np

def fft(x):
    n = len(x)
    if n == 1:
        return x
    even = fft(x[0::2])
    odd = fft(x[1::2])
    T = [np.exp(-2j * np.pi * k / n) * odd[k] for k in range(n // 2)]
    return [even[k] + T[k] for k in range(n // 2)] + [even[k] - T[k] for k in range(n // 2)]
\end{lstlisting}
The FFT is a widely used algorithm for computing the \textbf{Discrete Fourier Transform (DFT)} of a sequence. It reduces the number of computations needed from O(n$^2$) to O(n log n) 12. The basic idea of the FFT is to apply divide and conquer. We divide the coefficient vector of the polynomial into two vectors, recursively compute the DFT for each of them, and combine the results to compute the DFT of the complete polynomial 3.

I hope this helps!

\end{ReportAiUse}
\end{document}
%% 
%% This work consists of these files mcmthesis.dtx,
%%                                   figures/ and
%%                                   code/,
%% and the derived files             mcmthesis.cls,
%%                                   mcmthesis-demo.tex,
%%                                   README,
%%                                   LICENSE,
%%                                   mcmthesis.pdf and
%%                                   mcmthesis-demo.pdf.
%%
%% End of file `mcmthesis-demo.tex'.
